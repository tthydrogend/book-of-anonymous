\part{Одяг}
\chapter{Класичний одяг}
Посеред інших речей формальний одяг — це ключ виглядати гарно і розумно. На
співбесідах не виглядатимеш як не приший кобилі хвіст, елегантно дивитимешся
як на вечірках, так і в гостях у батьків твого трапа.

Передумовою є те, що ти повинен слідувати готовим правилам, а не вигадувати
нові. В іншому випадку я не бачу причин тобі це читати.

Деякі розділи глави застосовні не лише до класичного одягу. Вгадай які.

\section{Чоловіки}
На щастя для чоловіків, класичний одяг зводиться до костюму та сорочки, чи
додатково до смокінгу, проте це буває вкрай рідко.

\subsection{Сорочки}
Це вже обговорювалось тисячі разів. Сорочки потрібні і крапка.

Більш ніж ймовірно, що тобі треба турбуватись за колір і тканину більш за все,
так як вони найбільше кидаються в очі.

\subsubsection{Тканина і догляд за нею}
На сьогодні сорочки роблять з бавовни, льону чи шовку, деякі змішують
матеріали. Такі деталі є на бірках. Звертай на це увагу; більшість сорочок
можна прати в пральній машинці, але деякі (особливо шовк) потребують
спеціального догляду, май це на увазі.

Більшість є матовими (тобто не блискучими), проте бувають і в стилі атласних.
В загальному останнім більше пасують темні коштовні кольори, проте завжди є
виключення. Залежно від того як ти збираєшся їх носити треба відштовхуватись не
стільки від кольору, як від типу тканини.

\subsubsection{Кольори, узори і прикраси}
\paragraph{Колір.} Спектр кольорів надзвичайно широкий. Немає такого поняття як
„нормальний“ колір, хоч білі і чорні сорочки користуються більшою популярністю
через простоту кольорів. Загалом у твоєму гардеробі має бути щонайменше одна
біла сорочка.

Кілька головних правил:
\begin{itemize}
	\item[--] на формальних вечорах надавай перевагу однотонним кольорам;
	\item[--] світліші відтінки для темних костюмів і темніші для світлих:
	\begin{itemize}
		\item[--] між сорочкою і піджаком має бути візуально замітна різниця;
	\end{itemize}
	\item[--] рожева сорочка на рожевому чоловікові виглядає страшно;
	\item[--] уникай яскравих кольорів, таких як червоний, зелений і жовтий.
\end{itemize}

\subsubsection{Узори та прикраси}