\chapter{Приведення себе до культурного вигляду}
\section{Пристойно виглядаємо}
\subsection{Прийом душу}
Приймання душу є до біса суттєвою штукою. Це те, що збереже тебе свіжим без
зайвих турбот.

Необхідні речі: мило (згодиться як тверде, так і рідке), душ, рушник. Дійсно, це
все, що тобі необхідно. Не віриш? Тоді читай далі…

Додатково можеш скористатись: шампунем, кондиціонером, губкою /
фланеллю\footnote{Фланель — м’яка шерстяна чи бавовняна ворсиста тканина полотняного або
саржевого переплетіння, зазвичай з пухнастим рідким начосом. Іншими словами,
спеціальна ніжна тканина.} / щіткою для чухання спинки / скраб-рукавичкою.
Домашнє завдання: продовжити список речима, які застосовуються для ванних
процедур.

Етапи прийому душу:
\begin{enumerate}
	\item Ввімкни душ, полізай всередину.
	\item Сполоснись — перед використанням мила швидко обітрись, намочи
	голову — це змиє частину бруду.
	\item Помийся. Це також має бути очевидним. Починай зверху і рухайся вниз.
	Таким чином ти уникнеш бруду там, де ти його вже змив.
	\begin{enumerate}
		\item Почни з голови / волосся (підказка: використання шампуню
		опціональне, мило для цього цілком згодиться. Хоча, ти можеш обрати
		щось інше відповідно до власних витрат; зрештою, кондиціонери для
		метрофагів), натирай милом (чи будь-чим іншим) кілька хвилин, а тоді
		сполосни. Повторювати крок доти, доки не відчуєш голову чистою.
		\item Тепер лице. Знову ж таки, мило згодиться. Можеш використати
		фланель або скраб (скраб, не наждачний папір) для жирніших ділянок
		шкіри. Як і з волоссям, намилити / потерти / змити доти, доки не буде
		відчуття чистоти.
		\item Продовжуй спускатись вниз: руки, пахви, груди, спина, пеніс
		(це також включає залуплювання), яйця, статеві губи (постарайся, щоб
		мило не попало у піхву), сідниці, анус (хорошою ідеєю буде просратись
		перед прийняттям душу), ноги, ступні (включаючи між пальцями).
		\begin{itemize}
			\item[--] Якщо ти не обрізаний, не рекомендую мити головку з милом,
			так як це може зіпсувати звичний баланс мікроорганізмів і
			спричинити баланіт. Ну або як мінімум це може подразнювати шкіру.
			Писано з досвіду.
		\end{itemize}
		\item Для тіла будуть корисними абразиви — „зернисте“ мило чи губка /
		фланель / скраб, про які вже говорилось вище.
	\end{enumerate}
	\item Сполоснись (знову). Впевнись, що все мило і інше змите.
	\item Обсушись. Ретельно витрись рушником (чи двома, якщо ти бридишся
	використовувати те саме для лиця і дупи). Знову, почни від волосся і
	спускайся вниз до ніг. Зверни особливу увагу на ті зони, де шкіра треться
	об шкіру (попа, пахви тощо).
\end{enumerate}

Пам’ятай зняти з себе все те, в що ти вбирався в цей або вчорашній день. Це
означає, що незалежно чим ти пахнув, парфумами чи випивкою, після душу ти
повинен пахнути милом. Якщо ти не один з тих щасливчиків, у яких запах поту
невідчутний (знайди того, хто це підтвердить), вибери якийсь дезодорант чи
антиперспірант і його використовуй. Користуйся твердими / роликовими
антиперспірантами, так як спреї закінчуються швидше, а також тому, що покривати
своє тіло цією хімією є гірше, ніж смердіти потом. Акс і інші дезодоранти не
зваблюватимуть дівчат і вони не кидатимуться тобі на шию, це лише маркетинговий
хід, тому якщо тобі подобається цей запах, використовуй його в ПОМІРНІЙ
кількості!

\subsection{Гоління}
Наступне завдання, що стоїть перед тобою, це гоління. Якщо не можеш відростити
гарну бороду, тоді брийся. Запущена борода псує чудове обличчя; гарна борода
прикрашає страшне лице. Це не означає, що ти маєш відрощувати бороду: зазвичай
надається перевага гладко вибритим чоловікам. Єдине виключення — «триденна
щетина», яку окремі жінки вважають сексуальною. Але це те, що приходить з
часом, тому просто брийся щоранку.

Незалежно від того як ти будеш бритись, ось кілька порад щодо догляду за
шкірою:
\begin{itemize}
	\item[--] вода: нехай лице буде постійно вологим. Зволожуй чим бажаєш:
	водою, кремом для бриття, милом тощо. Таким чином леза краще бритимуть, а
	ти не перетвориш своє обличчя в місце проведення військових дій;
	\item[--] змінюй леза / бритву приблизно раз в тиждень з того розрахунку,
	що ти бритимешся щодня. Чи якщо ти гівноїд і купуєш одноразову херню,
	стань метрофагом і придбай оте дороге лайно, або щось середнє. Щоб шкіра
	була гладкою, леза мають бути гострими;
	\item[--] не брий проти напряму росту волосся, ти ж хочеш гарно побритись.
	Спершу зроби кілька рухів за напрямом щетини. Після того, як будеш вибритий
	і воно буде мати пристойний вигляд, можеш спробувати побритись проти росту
	волосся. Але знову, не треба брити прямо проти росту, рухай бритвою під
	кутом. Таким чином ти не лише гладко побриєшся, але й за одно не вб’єш леза
	бритви (щодо небезпечних бритв, тести не проводились);
	\item[--] знову ж таки, спробуй електробритву. Навіть якщо для твого
	обличчя вони менш ефективні, гоління електричною бритвою, а потім звичайною
	легше сприйметься шкірою, а також леза звичайної бритви довше проживуть.
	Не треба купувати ті стодоларові. Невеличкий трімер за \$25 підійде цілком
	і зменшить об’єм роботи для звичайної бритви;
	\item[--] також можеш побрити інші частини тіла (лобкове волосся (НЕ
	ГЕНІТАЛІЇ!!), волосся на дупі) і уникнути інфекцій, накопичення поту і
	чого ще.
\end{itemize}

Якщо за якимось жахливим збігом обставин ти приваблюєш жінок і вона надає
перевагу тобі вибритому, можеш подивитись в сторону небезпечних бритв:
\begin{itemize}
	\item[--] для цього є багато причин і головною з них це та, що ти можеш
	гострити леза, зберігаючи гроші від витрат на одноразові. Інший плюс —
	воно бриє ідеально, достатньо двох рухів для видалення волосся і мертвої
	шкіри. Ну і це естетично приємно користуватись такою бритвою;
	\item[--] плюси щодо корисності полягають в тому як працюють одноразові
	бритви. Твоя Mach 3 піднімає волосся і зрізає їх, інколи по кілька разів,
	якщо ти робиш кілька рухів. Тепер перемнож число рухів на три. Шкіра тоді
	заростає / гоїться біля волосяних фолікул. Вже розумієш про що я? І ти ще
	дивуєшся чому леза тупляться, а волосся росте як дідько. Користуючись
	небезпечною бритвою в тебе не буде цих проблем. Але якщо ти будеш
	недостатньо обережним, то будеш виглядати ніби Росомаха чи Фредді Крюгер
	сплутали тебе з точильним каменем.
\end{itemize}

Якщо ж ти дуже лінивий, можеш піти в перукарню. В ту, що з гребінцями, синіми
накидками, перукарним столиком збоку і дідами з шахами в кутку. Ці жінки знають
як побрити хлопця швидко і правильно. Буде все — гарячі рушники, піна, лосьйон
після гоління, все як треба. Це займе п’ять-десять хвилин і сама процедура
дуже таки заспокійлива. Спробуй якщо не віриш.

\subsection{Чищення зубів}
Прийшов час до тієї обов’язкової штуки, якої вчила тебе твоя мама — чищення
зубів. Чищення зубів, так як і використання зубної нитки, є важливим етапом.
Перед тим як виходити з хати почисти зуби і точно вже почисти зуби після
сніданку. Перевір язик на наявність гною (загалом, це треба робити кожного
разу коли опиняєшся у ванній з дзеркалом), почисти зубною щіткою або зішкрябай
нігтем (понюхай його після, ти дійсно хочеш тримати оце в роті? Тепер помий
руки). Якщо ти збираєшся говорити з людьми і думаєш, що в тебе з рота може
пахнути, жуй гумку. Носи з собою жуйку для таких випадків.

Інша річ, про яку постійно забувають, це зубна нитка. Oral-B «Hummingbird» і
зубна нитка допоможуть дістати до складнодоступних місць зубів. Ти можеш сидіти
перед телевізором чи комп’ютером під час цього процесу, тут можна обійтись
без дзеркала. Масажери ясен, вибілюючі смужки і інше схоже лайно йдуть лісом
доти, доки ти щонайменше двічі в день чистиш зуби і вичищуєш зі
складнодоступних місць раз в день. Хоча, Безосібний без хорошої основи
ротової гігієни виявити для себе дуже зручними вибілюючі смужки.

Використання рідини для полоскання роту є вирішальним для твого поганого запаху
з рота. НЕ ВИКОРИСТОВУЙ рідину, яка містить в собі дезинфікуючі речовини. Вони
вимивають природні бактерії. Такі рідини для полоскання лише маскують запах
на короткий час, в довшій перспективі запах лише погіршиться. Якщо хочеш
вибрати щось хороше, вибери щось на кшталт Breath RX. Не турбуйся за смак,
раніше чи пізніше ти до нього звикнеш. Чесно кажучи, він покращить твоє дихання
вдесятеро, так що зроби нам всім послугу і користуйся ним.

Пам’ятай про щітку для язика. Після чищення зубів почисти язик і тоді сполосни
ротову порожнину хорошою рідиною для полоскання. Коли чистимеш язик, постарайся
досягнути якомога далі назад і вичищуй назовні. Більшість лайна, що спричиняють
поганий запах з рота, це гнила їжа, що застряла на язику ближче до горла.

Останнє, також важливе: причісуй волосся. Причісуватись треба незалежно від
довжини волосся. Для тих, хто має довге, причісування надасть гарнішого і
впорядкованішого вигляду. Для тих, в кого воно коротке, воно не виглядатиме як
тотальний хаос. Після цього можна щось робити з волоссям (гель для волосся,
фіксатор тощо). Лише не перестарайся — якщо ти витрачаєш стільки ж часу,
скільки й твоя дівчина, це вже проблема. Альтернативно, якщо ти виглядаєш так,
ніби щойно виліз з душу, нагелив волосся і так і пішов, не турбуйся. Ти нікого
не обдуриш і всі навколо спідтишка над тобою насміхатимуться. Пам’ятай: вміння
набувається з часом і досвідом.

\subsection{Справляємося з прищами}
Прищі як похмілля. Всі їх мали і в загальному вони всі лайно. Однак, ось кілька
базових підказок як перетворити лице з піци пепероні в сирну піцу. Або з
шоколадно-полуничної піци в просто шоколадну, якщо ти негр.

Якщо у тебе насправді погано з вугрями, то тобі краще сходити до дерматолога.
Запитай його, він назначить різні креми для різних ступенів важкості. Якщо ж
у тебе більш-менш нормально, то можеш справитись і сам відповідними кремами.
Прищі з’являютьс ятоді, коли пори забиваються мертвою шкірою, а тоді
заражуються бактеріями. Коли купуєш миючий засіб для лиця, подивись чи він має
саліцилову кислоту серед активних компонентів. Саліцилова кислота —
відшаровуючий агент — цікаво, що він знаходиться в аспірині, в крайньому
випадку з якого можеш зробити ексфоліант\footnote{Косметичний засіб чи агент
у складі засобу для швидкого відлущування зовнішнього шару епідермісу.}. Воно
запобігає забиванню пор. Далі, придбай якийсь крем проти вугрів, який містить
в собі пероксид бензолу. Він вбиває бактерії, які розводяться в закупорених
порах. Мий лице і використовуй крем два-три рази в день. Проте починай з одного
разу в день щоб впевнитись що в тебе немає алергічної реакції. Якщо крем сильно
сушить шкіру, використовуй зволожувач. Лише впевнись, що зволожувач не заб’є
тобі пори, інакше ти замкнеш коло.

Інакше, щоб уникнути вугрів, спробуй м’яке мило (дитячі мила) або щось
на кшталт антибактеріальне. Мий лице двічі в день: як лише прокинувся і перед
сном. Твоє лице жирне за замовчуванням, так що за це не турбуйся, але якщо це
починає бути замітним на подушці, значить лице треба мити частіше. Якщо бачиш
що десь назріває вугор, візьми краплю крему проти вугрів і втри в нього. Воно
через деякий час має пройти. Врешті-решт, крапля профілактики вартує кілограму
ліків. Важливо не перестаратись зі скрабами і миючими засобами. Колись
безперестанку знущався над дитиною? Сумнівно, але якщо так, тоді ти знаєш, що
одного дня він прийде в школу зі зброєю і нагне тебе в колумбійському стилі.
Те ж і з лицем. Не зловживай з хімією, так ти лише зробиш гірше.

\section{Менш часті запитання}
\subsection{Обріж і зістрижи все зайве}
Досить бути неотесаним тролем — підстрижи нігті на руках і на ногах, помий
вуха, вистрижи волосся з ніздрів, з родимок, приведи в порядок брови, зістрижи
задирки. В додаток збрий свій кущ. На голому тілі член виглядатиме більшим, а
також менше шансів, що ти злякаєш дівчинку власними джунглями. Менше волосся,
більше тіла.

\subsection{Догляд за шкірою}

Хоч це може звучати по-дівочому, треба слідкувати за здоров’ям шкіри. Я не
говорю про похід у спа-салон кожного тижня і манікюр щовівторка, але
мінімальний догляд не зашкодить. Це також зменшить обсипання. Якщо маєш великі
родимки чи бородавки, то сходи до дерматолога. Можливо, їх можна буде видалити
(деякі можуть бути раковими і мають бути видалені). Грибок на ногах то жахливо
противна і досить незручна штука, для запобігання користуйся кремом Tinactin.
Плями на шкірі також мало приваблюють і вимагають як мінімум деякої уваги. Не
треба бути фріком, але не будь тролем. Більшість необхідних речей знайдеш в
аптеці. Запитай продавця, якщо щось не можеш знайти.

\subsection{Гоління — хлопці}
Це постійна тема для обговорень в онлайні. Ти, являючись чоловіком, ймовірно
схильний до відрощування волосся на тілі. Деяким дівчатам це подобається,
деяким ні. Все залежить від тебе, чи ти хочеш виглядати як медвідь чи ні.

Віриш чи ні, більшості дівчат абсолютно паралельно чи в тебе волосся як
в горили (наприклад, на ногах). Тим не менше, якщо у тебе гарне тіло, було б
добре його збрити. Проте тебе можуть висміяти, тому обережно.

Живіт — інше місце для «обрання власного шляху». Якщо в тебе там не килимове
покриття, тоді для його збривання мало причин. Зрештою, якщо в тебе небагато
„пухнастого“ волосся, яке виглядає по-ідіотськи, збривай його сміливо.

Волосся на спині непривабливе і може бути неприємним, так що ніхто не буде
дорікати, якщо ти його повністю збриєш.

Лобок треба підстригати. Завжди. Так, це може видаватись дивним чи тобі може
прийдеться робити це приховано, але це також квиток в жіночий рот і в інші
дірки зокрема. Це не означає, що треба голити все начисто (хоча це гарно
виглядає), просто воно має бути зістрижене. А не срані джунглі. До того ж, це
таокж буде плюсом для тебе — ніякого більше лобкового волосся у ванній на
підлозі, що, до речі, виглядає гидотно.

\subsubsection{Попа}
Заради бога, брий дупу. Ніхто не хоче то бачити. Але під час цього безсумнівно
кумедного процесу будь обмережний. Прислухайся до старої копіпасти:

\begin{quote}
Нещодавно я зробив помилку в житті і даю тобі цю історію, щоб ти її міг
уникнути. Все почалось, як і багато чого починається, з того, що я не міг
прокакатись. Ні, в мене не було запору; проблема полягала не в регулярності,
а в техніці. Здавалось, волосся на попі виросло до такої степені, що крихітні
шматочки лайна постйіно застрягали у волосяних джунглях між сідницями. Це
спричиняло шалену фрустрацію, так як я ЗНАВ що ще трохи має випасти, але був
не в змозі струсити чіпке лайно з його волосяного місця проживання. Врешті-решт
мені залишаллись дві речі: вишкрібати туалетним папером непрошеного гостя
(що б вимагало великої точності для уникнення розмазування створіння по заду,
особливо з того погляду, що я взагалі не бачу що я роблю), або ж просто піти
у ва-банк, почати витирати і сподіватись що вдасться витерти всі фекалії до
того, як закінчиться туалетний папір. Я дивився на проблему, коли мені раптово
прийшла геніальна на той момент ідея. Я подумав: «ей, це моя дупа і моє волосся
на ній, правильно? Тоді чому б не збрити його, тоді ж все лайно вивалиться з
мене як пиво тече з бочки!». Це одне з тих тверджень, які попадуть в історію,
про які жалкуватимуть. «Чи багато ж там може бути індійців?» — сказав генерал
Кастер. «Хорочий день для їзди» — JFK. «Нарешті! Америка он-лайн має повний
доступ до Usenet!» — сказав якийсь ідіот з техпідтримки. Такою ж була моя ідея
вибрити дупу. Я виконав бойову операцію цієї ж ночі, дешевим станком і сидячи
на рушнику. Рухаючись від внутрішньої частини половинок назовні, я почав важкий
процес звільнення попи від волосся. Іноді приходилось скидувати волосся з леза,
що я робив витираючи станок об рушник. Повільно мої близнята і впадина між ними
ставали схожими на ніжні голі рожеві ягодиці новонародженого малюка. Нарешті
я стер станок востаннє і оглянув пророблену роботу. Рушник був вкритий купкою
волосся. Моя попа була гладка як слонова кістка. Я посміхався, я був
задоволений, думаючи що моїм проблемам кінець. Як мало я знав! Тепер я розумію
навіщо на попі росте волосся. Як і все, що бог створив у цьому світі, навіть
воно мало своє призначення. Лише після того, як я його збрив, я зрозумів яким
воно було важливим і як я сприймав його як належне. По-перше, воно забезпечує
тертя. Це я дізнався на наступний день, коли прямуючи до класу вийшов на сонце.
Після піднімання двома прольотами сходів і починаючи потіти, я відчув щось
неприємне. Піт збирався між половинками і спричиняв неприємні відчуття з кожним
кроком від їхнього взаємного ковзання. Мене не покидала думка про те, щоб піти
в душ і витерти задницю досуха, але мені було треба йти в клас. Зрештою, я
сподівався, що піт висохне. На жаль, він висох, але лише після того як змішався
з молекулами лайна, які затримались навколо моєї коричневої зірочки. Коли я
увійшов до класу, ягодиці злиплись липкою лайняно-потовим шейком. По дорозі
додому попа почала свербіти. Тисяча чортів, воно свербіло! Таке відчуття, наче
рій мурах ходили між булками. Борючись з бажанням запхати туди руку і
почухатись, я поквапився додому. На жаль, від цього напруження я ще більше
спітнів, так що коли я нарешті добрався до своєї кімнати, половинки літали
вверх і вниз. Я швидко скинув штани і спробував висушити попу феном, розсунувши
ягодиці. Треба розуміти, що без волосся близнятка щільно прилягають і можуть
утворювати вакуум. Результатом був пердіж, який вирвався з попи як
передсмертний крик. І якщо цього недостатньо, в мене є чудове продовження
драматичних тортур. Будь-хто, хто що-небудь брив, знає, коли починає рости
волосся, воно як щетина. Уявіть собі попу з текстурою щетини лобкового волосся.
Ну, це те, що в мене зараз є. Це пекельна тортура, я вже стільки разів дивився
у вікно і думав що тримає мене від того, щоб вистрибнути і покінчити з цим
всім, аніж терпіти постійну агонію. Друзі, НЕ БРИЙТЕ ВОЛОССЯ НА ПОПІ!
\end{quote}

Ця копіпаста — чудовий приклад підводних каменів бриття попи. Тим не менше,
прийнятно стригти, або ж навіть збривати волосся між дупою і яйцями, якщо
воно свербить чи на ньому збирається багато бруду. Підстригання волосся на
дупі в нормі доти, доки воно не надто близько до ніжних місць, інакше вийде
так само як в копіпасті. Підстригання має запобігти накопиченню гівна і
сверблячої і пахнючої штуки, про яку ти щойно читав.

\subsection{Гоління — дівчата}
Як вже говорилось, тобі вибирати як тобі виглядати. Мабуть, неголене тіло
також оцінять.

Особисто я згоден з цим\footnote{Треба розуміти, йдеться мова про те, що
розповідач згоден з неголеністю дівчат. Щодо мене, гола шкіра виглядає
сексуальніше.}, але я один з шести мільярдів; ніхто не вимагає мати шкіру як
в дитини, але більшість зменшують кількість волосся до мінімуму. Всього лиш
питання смаків. Також пам’ятай, коли будеш про це думати, що чим менше у тебе
волосся, тим ймовірніше, що тобі вилижуть між ногами. Особисто я люблю пахви
і волосся під пахвами стримує від того, щоб їх лизати, те ж і для піхви.
Можливо це через те, що ми залежні від порно, і тому що ми хочемо бачити
солодке, а не кущ, який їх приховує. Здається, що жінкам тріммер приносить
неабияку насолоду. Пам’ятаю історію про хлопця, який стриг дівчатам тріммером,
тим самим збуджуючи їх, щоб насолодитися їхнім соком.

Аналогічно хлопцям, це питання гігієни. Чим більше волосся, тим складніше його
мити і тим ймовірніше, що в ньому збиратиметься піт і бактерії. Тріммер —
нормально, бриття — категорія екстра, депіляція напевно що болюча.

Тут треба жіноча думка.

Ще одне про бриття. Якщо ти нічого не робиш зі своїм волоссям, не очікуй, що
хлопець буде щось робити зі своїм.

\subsection{Засмага}
Засмага, осуджена більшістю чоловіків, зазвичай є білетом від товстого
мішка з лайном до гарного мішка з лайном. Будь з собою відвертим — дівчата
зі засмагою, особливо голі дівчата зі засмагою, виглядають до біса привабливо.
Як висновок, ти зі засмагою, ймовірно, також виглядатимеш гарно.

Згоден, це трохи дивне твердження, але це правда. Багато порнозірок спеціально
засмагають, тому що вони знають, що голі форми солодше виглядають з краплею
засмаги. Іншими словами, легка засмага цінується більше, ніж біла шкіра. Якщо
ти не дрочиш на японців / «Середні століття™», тоді тобі треба як мінімум
спробувати збільшити рівень пігментів.

«Навіщо?» — можуть запитати негативно налаштовані. Згадай предків: печерні
жителі $\rightarrow$ білі, аль-факуни\footnote{Від альфа-кун. Кун — японське
«чоловік» (чи щось таке).} $\rightarrow$ засмаглі.
% Now tell me, how is babby formed? — в контексті перекладеного не розумію
% до чого це тут чи як це перекласти.

Засмагання чудово підходить до басейну. Сходи у відкритий басейн\footnote{Той,
що під відкритим небом. Те, що басейн має працювати, зрозуміло і без того.} і
просто полежи на лежаку. Поплавай. Відпочинь. Знову поплавай. Це приємно,
розслабляюче, а також ти стаєш привабливішим.

Також засмагання попереджує вищезгадані проблеми з вугрями, зокрема на спині
і грудях. Зауваж, якщо в тебе все зовсім погано, не засмагай і проводь менше
часу на сонці, інакше можеш зробити лише гірше. Якщо є додаткові питання,
запитай дерматолога. Плавання очищує шкіру.

Чорні або інші раси з натуральною темною шкірою мали це на увазі. Навіть якщо
вони будуть проживати в басейні, від цього не буде толку. Хоча сам по собі
басейн то здорова штука.

Само собою розуміється, що не можна засмагати до чорного / яскраво-
помаранчевого / жареного, інакше ти випишеш собі смертний вирок, і я не жартую.

\subsection{Стрижка}
Стрижка, на кшталт одягу, задає хто ти є. Вона може бути вдалою і невдалою.
Переважно ти або вдало стрижешся, або все псуєш. З правильною стрижкою ти
класно виглядатимеш і в найгіршій ситуації, коли зіжахливою стрижкою ти
виглядатимеш так, ніби хочеш трахнути троюрідну сестру (хоча, ніби для тебе
це є щось погане). Хоча чітких рамок для визначення хорошої стрижки немає,
пам’ятай наступні речі:

\begin{itemize}
	\item[--] в загальному короткі стрижки жахливі. Під короткою стрижкою
	я маю на увазі „скінхедські“ стрижки, з якими більшості чоловіків треба
	вмерти і горіти в пеклі;
	\item[--] знай послідовність:
	\begin{enumerate}
		\item Спершу знайди для себе салон. Саме салон, а не перукарню. В
		салонах зазвичай працюють жінки, що означає, що вони більше знають про
		стиль і бачили зачіски не лише в новинах\footnote{Підозрюю, що це
		country-related. В мене в місті всюди перукарні, людей стрижуть
		чудово. Я також хожу в перукарню. Мама каже, що в мене чудова
		зачіска.}.
		\item Запишись і прийди вчасно. І заради бога, ЗАПИТАЙ СКІЛЬКИ ПЕРУКАР
		ПРОПРАЦЮВАВ. Ніколи не давай підсунути собі „новеньку“ (а вони
		захочуть).
		\item Якщо тобі все-таки підсунули, можеш не сумніватись що тобі
		зіпсують зачіску жахливим почуттям прекрасного. Це те саме, якби ти
		замовив у повара приготувати те, що він раніше ніколи не готував.
	\end{enumerate}
	\item[--] стрижка „під їжачка“ невдала, дивись вище;
	\item[--] тим не менше, якщо в тебе вся шкіра в вугрях, стрижись накоротко;
	\item[--] якщо тобі подобається стрижка „під машинку“, СТРИЖИСЬ САМ;
	одноразово витраться на тріммер, буде дешевше і швидше;
	\item[--] стрижка під афро пасуватиме тобі лише в тому випадку, якщо ти
	чорний. І носиш костюм. Бажано, надоїдаючи іншим, возячись навколо
	басейну;
	\item[--] навіть якщо ти відрощуєш волосся, стригтись треба. Знайди
	хорошого стиліста, він знає що робити;
	\item[--] жирне довге волосся — індикатор того що ти дрочиш ночами на
	фуррі і бавишся D\&D;
	\item[--] щодо стилю, найкраще питати стилістів що тобі пасуватиме. Задай
	таке ж питання, як і при купівлі одягу: «Що, на Вашу думку, мені підійде?»
	чи «Якби Ви йшли зі мною на побачення, як я мав би виглядати?». Діє
	чаруюче, але не в тому випадку, коли стиліст обдовбаний псих. Один
	безосібний дуже швидко зрозумів, що в маленькому містечку зі сумнівного
	виду дівчатами таку фразу краще не використовувати.
\end{itemize}
Висновок: навіть якщо в тебе найкраща у світі зачіска, лайняний стиліст зіпсує
її в секунді, намагаючись бути „оригінальним“ чи, точніше, „тупим ідіотом“.

Як знайдеш для себе стрижку, чітко поясни перукарю що ти від нього хочеш. Не
соромся показати фотографію обраної стрижки, якщо не можеш пояснити на словах.
Більшість перукарів дотримуватимуться твоїх вказівок і ти ймовірно отримаєш те,
чого й хотів. Натомість, залишивши перукаря з туманними вказівками, результат
може тебе розчарувати.

Давай складемо список як НЕ ТРЕБА стригтись. \textbf{ЗАБОРОНЕНІ СТРИЖКИ:}
\begin{itemize}
	\item[--] „їжачок“;
	\item[--] „під шапочку“ (голова стає схожа на головку);
	\item[--] косіння під Бітлів;
	\item[--] стрижки, які вимагають агресивного випрямлення волосся;
	\item[--] стрижки, які вимагають помаду (що?);
	\item[--] „триденна щетина“ на голові;
	\item[--] малет\footnote{Тип стрижки, де волосся підстрижені коротко
	спереду і по боках, а позаду залишаються довгими. Ґуґлити „mallet“.};
	\item[--] хвостики, особливо жирні;
	\item[--] „свиняче\footnote{Грубе волосся на лиці в районі вус, підборіддя
	і скул. Такі зазвичай носять придурки, які застрягли в 50-х, або лисіючими
	хренами, які хоч якось хочуть це компенсувати.}“ волосся;
	\item[--] будь-що, що вимагає яскравих кольорів (зелений, червоний, голубий
	тощо);
	\item[--] відбілювання блондинного жовтого кольору;
	\item[--] фарбування частини волосся (наприклад, чілок);
	\item[--] будь-які зачіски під „емо“.
\end{itemize}

\subsubsection{Фарбування}
Якщо ти з якихось невідомих причин хочеш пофарбувати волосся, пам’ятай
наступне:
\begin{itemize}
	\item[--] надавай перевагу натуральним кольорам чи їх варіаціям;
	\item[--] темні кольори виглядають звичніше і натуральніше;
	\begin{itemize}
		\item[--] незважаючи на це, оминай емочорний;
	\end{itemize}
	\item[--] зійдуть навіть окремі кольорові смужки, доки їх колір мало
	відрізняється від головного;
	\item[--] дешева фарба шкодить волоссю;
	\item[--] якщо не впевнений в результаті, пофарбуй лише кінчики. В разі
	чого ти їх зможеш зістригти без шкоди зачісці.
\end{itemize}