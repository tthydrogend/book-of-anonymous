\setcounter{page}{2}
\section*{Про книгу}
\textbf{Переклад книги «The Well Cultured Anonymous».}
«Повернення культурного Безосібного» — це оновлена книга, що базується на
оригінальній «Культурний Безосібний», яка була ретельно скомпонована другом
Безосібним на Вікічані. Вона намагається показати (в основному іншим
безосібним) як бути розумним, культурним та ввічливим у сучасному світі.

Необхідно оновити деякі частини, додати важливі теми і підлаштувати кілька
сторінок під особливості жіночої частини аудиторії. Якщо бажаєш допомогти,
нижче будуть подані вказівки як це можна зробити.

Ми додали «Повернення» до назви книги, так як після розвалу Вікічану оригінал
був втрачений. Кілька розділів все ще живуть на Культурному — спамерському
сайтику, розробленому колишнім адміном Вікічану у невдалій стробі створити
лайняний чоловічий журнальчик. Через таку несправедливу долю для такого
інформативного гайду ми відновлюємо оригінал та відкриваємо його для редагувань.

\section*{Інтро}
Ви їх бачили всюди — топіки самодопомоги з вимолюванням різних порад. Одні не
мають уяви що можна вдягнути окрім футболки з пантерою і джинсів, які їм
купила мама п’ять років назад. Інші не знають як підкотити яйця до жінок і
гають час фапанням\footnote{Фап (fap) — дрочка, мастурбація.} на лолі. А хтось
лише хоче дізнатись чи треба брити лобок.

Ось, друзі, відповідь. Написана безосібним, редагована безосібним і відточена
безосібним. Всі хитрості і бочки, які тобі стануть у пригоді. Перші глави
цієї книги націлені висвітлити і нагадати основи — як пристойно виглядати,
як залишатись у формі і таке. Далі ми перетечемо до теми відносин: як поводитись
з жінками, як спілкуватись з іншими людьми і що робити на вечірках. Після
розглядатимуться мішані теми — в загальному все інше, що ти хотів знати.

Читай. Вчись, Безосібний.

\section*{Дисклеймер}
Подану інформацію не треба сприймати буквально чи використовувати для
творення беззаконня. Вона надається лише для ознайомлення та ніяк не
спонукає до того, про що там пише. Ти береш на себе всю відповідальність за
все, чим збираєшся скористатись.

\newpage
\tableofcontents

\chapter{Про книгу}
Тому що це робота багатьох людей.

\section{Автори оригіналу}
Наступний список — це автори, редактори і помічники, які зробили свій вклад
в оригінал «The Well-Cultured Anonymous». Ці люди тим чи іншим чином
сприяли появі цієї книги, приклали зусиль для збору інформації. Дякуючи їм
ми її маємо.

\begin{multicols}{3}
\begin{itemize}
	\item[--] Achan
	\item[--] Andreas
	\item[--] Anonymous
	\item[--] Aphextwin
	\item[--] Appelation
	\item[--] Awkner
	\item[--] Berserker
	\item[--] Bluith
	\item[--] Cosa Nostra
	\item[--] Copypasta Baker
	\item[--] Cyber Pope
	\item[--] Danguy
	\item[--] DerWelt
	\item[--] Eaglewolf
	\item[--] Femme Fatale
	\item[--] FH Regulus
	\item[--] Frostleaf
	\item[--] GG
	\item[--] Harbl
	\item[--] HowIShoopWoop?
	\item[--] Java378
	\item[--] Kuroboushi
	\item[--] Laniac67
	\item[--] masslac
	\item[--] MasterChief-117
	\item[--] Mikey
	\item[--] mreddy1
	\item[--] Nami
	\item[--] Nessunome
	\item[--] Ninja337
	\item[--] Orion
	\item[--] Over9000
	\item[--] Raccoon
	\item[--] Ras-chan
	\item[--] RKERONESKE
	\item[--] RoboChocobo
	\item[--] Sarafan
	\item[--] ScenesFromAMemory
	\item[--] Selentic
	\item[--] SonicDragonwerkz
	\item[--] Triple Chan Soul
	\item[--] VoodooKobra
	\item[--] WikiSysop
	\item[--] Yum22Yum23
	\item[--] Zeph
\end{itemize}
\end{multicols}

\section{Перекладачі}
Без них не було б української версії. До речі, вчи іноземну мову, пригодиться.
Список перекладачів поповнюватиметься та розширюватиметься. Якщо список
розростеться надто сильно, глава піде в кінець книги.