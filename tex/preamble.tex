\documentclass[a4paper,oneside]{book}
\usepackage[utf8]{inputenc} % задати кодування для бейбела
\usepackage[ukrainian]{babel}
\usepackage{indentfirst} % робити відступ першому абзацу
\usepackage[pdftex]{graphicx} % дозволити вставку зображень
\usepackage{fancyhdr} % колонтитули
\usepackage{titlesec} % представлення виводу назв рівнів секцій
\usepackage{hyperref} % створити клікабельні посилання
\usepackage[left=1cm,right=1cm,top=2cm,bottom=3cm,bindingoffset=0cm]{geometry}
\usepackage[labelsep=period]{caption} % Рис. НОМЕР{.} Опис. - додає оту крапочку
\usepackage{tabularx}
\usepackage{tocloft}
\usepackage{multicol}

\let \savenumberline \numberline
\def \numberline#1{\savenumberline{#1.}}
\renewcommand{\cftchapleader}{\cftdotfill{\cftdotsep}} % put dots for chapters
% \setlength{\cftchapnumwidth}{0pt} % прибрати сраний відступ у другій лінійці
% при переносі розділів

\addto\captionsukrainian{\renewcommand{\contentsname}{ЗМІСТ}}
\pagestyle{fancy} % стиль колонтитулів

\setlength{\parindent}{1.0cm} % відступ абзаців
\titlelabel{\thetitle. } % цифра. Назва
\hypersetup{ % задати кольоризації в посиланнях
colorlinks,
citecolor=black,
filecolor=black,
linkcolor=black,
urlcolor=black
}
